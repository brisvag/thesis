\chapter{Visualisation, annotation and analysis software}

In most scientific fields, the ability to visualize and interactively explore one's data is crucial to forming an understading of the analysed system, and for hypothesis generation.
Fields sporting data with higher dimensionality, such as cryo-ET, benefit especially from n-dimensional and interactive visualisation, as static plots and images are often unfit to capture the full scope of the system.

Unfortunately, as cryo-et workflows become more and more automated and software tools become more abstracted from the data itelf, it is easy --- and tempting --- to follow entire processing workflows while rarely looking at the data other than through summaries or final results.
This is often encouraged by the monolithic software suites currently dominating the field, making it harder to implement and use custom tools to inspect or intervene at any one point of the workflow.

TODO: metadata...

This often leads to 


\begin{outline}
\1 importance of visualisation (human-in-the-loop)
    \2 issues caused by inability to look at data (wasted work, harder to form hypotheses)
    \2 importance of interactivity (not just looking, but picking, annotating, etc). Automation is great when it works, but in cryoet it's tricky and it often doesn't
        \3 problem of automation vs customizability
\1 denoising and "prettifying"
    \2 usefulness for picking/intepreting
    \2 pitfalls (especially with rise of AI)
\1 annotation, segmentation, picking
    \2 especially hard in 3D, requires extra attention both in automated and manual procedures, and benefits especially from human-in-the-loop approach
    \2 existing software (not sure to what extent i should go here, or say to refer to the blik paper chapter)
\1 somewhere (maybe not here?) there should be a part on algorithmic/computational limitations, which are being or could be tackled in the future
\end{outline}
