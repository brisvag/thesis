\chapter{Visualisation, annotation and analysis software}

\begin{outline}
\1 importance of visualisation (human-in-the-loop)
    \2 issues caused by inability to look at data (wasted work, harder to form hypotheses)
    \2 importance of interactivity (not just looking, but picking, annotating, etc). Automation is great when it works, but in cryoet it's tricky and it often doesn't
        \3 problem of automation vs customizability
\1 denoising and "prettifying"
    \2 usefulness for picking/intepreting
    \2 pitfalls (especially with rise of AI)
\1 annotation, segmentation, picking
    \2 especially hard in 3D, requires extra attention both in automated and manual procedures, and benefits especially from human-in-the-loop approach
    \2 existing software (not sure to what extent i should go here, or say to refer to the blik paper chapter)
\1 section (or chapter?) on pipepine (and maybe waretomo)
    \2 some suites exist that do the "whole" procedure, but they are monolithic and hard to get in/out for special cases, which occurs frequently in cryoet
    \2 difficulties with metadata wrangling
    \2 ...

\1 I'd like to have a general section about software practices, but not sure where
    \1 a case for user-friendliness (cryosparc > relion)
    \1 a case for open-source (relion > cryosparc)
    \1 a case for domain agnosticism (napari)
    \1 the benefits of modularity, "clean code", documentation, etc, to the research community

\end{outline}
