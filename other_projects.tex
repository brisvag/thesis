\chapter{Other projects and tools}\label{other_projects}

This chapter presents a few other projects and tools that I worked on during my thesis, which are worthy of mention but do not necessarily belong to the main thesis body.

\section{Bacterial chromatin and HU}

Initially the main project of my PhD work, the study of bacterial chromatin compaction by cryo-EM and cryo-ET is an ongoing project in the group, building upon the work done during my first year.
The main protein of interest is HU, a nucleoid-associated protein (NAP) present in many bacteria (including \textit{E. coli} and \textit{D. radiodurans})...

\section{Teamtomo}
<`0:TARGET`>

\begin{outline}
\1 HU...
    \2 initial proposition: bacterial chromatin, connect to Dr chapter
    \2 interesting spiral formations, unexplained, reproducible
    \2 tricky sample preparation, very small protein
        \3 attempts as solving using DNA as scaffolding for picking, etc...
        \3 cryoet proof of concept
\1 teamtomo!
    \2 goal proposition, community efforts
\1 projection matching for eymeline's work
    \2 more native than SPA but more straightforward than STA
    \2 would be cool to go into some technical details here (cross-correlation, projection, etc)
    \2 optimizing for space and performance
    \2 the issues around cross correlation
\1 cs2star and cryosparc in general and related tools
\1 stemia and collection of tools... how to wrap this us in a way that makes sense?
\1 waretomo
    \2 some suites exist that do the "whole" procedure, but they are monolithic and hard to get in/out for special cases, which occurs frequently in cryoet
    \2 difficulties with metadata wrangling
    \2 ...
\1 molecular viewer napari
\1 would be cool to link here somehow to my CTF simulator in napari... Though probably not worth discussing CTF so in depth that we explain fourier transforms and how they combine to form an image?)
\end{outline}
