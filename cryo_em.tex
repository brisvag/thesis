\chapter{Cryo-EM and cryo-ET}

In this chapter, I describe the role of cryo-EM and cryo-ET in the field of structural biology, explain their fundamentals, and discuss advatages and limitations.

\section{Structural biology and X-ray crystallography}

Structural biologists employ several techinques to understand the structure and function of biomolecular systems, typically proteins, nucleic acids and their ligands. Arguably, the three giants in the field are X-ray crystallography, Nuclear Magnetic Resonance (NMR) and cryo-electron microscopy (cryo-EM), with several other techniques playing complementary, more specialized or niche roles (such as mass spectroscopy, neutron scattering or small-angle scattering). Each technique has pros and cons, making it suited to different samples and applications.

In X-Ray crystallography --- in many ways the predecessor to cryo-EM as all-purpose structural biology technique --- crystals are grown from the sample of interest; the crystals are then illuminated by an X-ray beam, and the diffraction pattern thus created can be detected and used to reconstruct the three-dimensional (3D) structure of the sample. Thanks to the short wavelength of X-rays --- as opposed to visible light --- it is possible to localize the positions of atoms with sub-angstrom precision CITE.
X-ray crystallography has played a major role in the development of structural biology, and is still the primary source of protein structures on the Protein Data Bank~\cite{bermanProteinDataBank2000,bermanAnnouncingWorldwideProtein2003}. However, the need for crystallization poses a few limits. The crystallization procedure is often hard to devise and reproduce, extending the time and resources needed for sample preparation. Moreover, the resulting crystal "forces" the sample into a crystalline lattice, often with biologically irrelevant chemical states, restricting the conformational freedom of the molecule and limiting the biological significance of the obtained structures CITE.

\section{Cryo-electron microscopy and single particle analysis}

Cryo-EM improves on these aspects, by using vitrification to capture the sample at a near-native state, and by forgoing crystallization and the analysis of diffraction patterns in favor of free single molecules and the use of the central slice theorem to combine many projections into a single 3D structure (known as single particle analysis or SPA).

\subsection{Base concepts and typical workflow}

(TODO: these base-concepts paragraphs here feel a bit out of place, I'm struggling to find where they should be and in what order...)

In simple terms, the electron microscope works by shooting a coherent electron beam at the sample, and using a camera to detect the scattered electrons and form an image. Thanks to the small wavelength of electrons, cryo-EM can reach much higher resolution than light microscopy; however, it presents unique challenges and limitations that its more familiar light-based counterpart does not.

In the last decade, the development of direct electron detectors made it possible to reach atomic resolutions, jump-starting to the so-called resolution revolution and the rise of cryo-EM as one of the primary methods for high resolution structure determination~\cite{faruqiCCDDetectorsHighresolution2000}.

Differently from light-microscopy, which relies on amplitude contrast for image formation, the electron microscope uses phase contrast, which is caused by elastic scattering events affecting the electrons traversing the sample CITE. Some electrons, however, are inelastically scattered; these electrons are no longer coherent, and therefore add to the noise of the image. Increasing the electron dose can help improve the signal-to-noise ratio (SNR), but comes at the cost of radiation damage, which denatures the sample and rapidly destroy high-resolution information. For these reasons, a significant limitation --- and thus optimization target --- of cryo-EM is the low SNR.

The cryo-EM workflow is well established, and usually consists of the same principal components: sample preparation and vitrification, data collection, preprocessing (cleaning, motion and CTF correction), particle picking and classification, three-dimensional (3D) reconstruction, and model building. This section describes such typical workflow, expanding on the theoretical bases surrounding each step.

\subsubsection{Sample preparation}
Cryo-EM samples for SPA are usually prepared in vitro by expressing ad purifying the protein or complex of interest to create a minimal system. Sample concentration and purity are important variables to control, as they will affect vitrification and data processing complexity. The sample solution is then deposited on a cryo-em grid and vitrified via plunge freezing or other methods (TODO: elaborate or just leave to citation?) CITE. Vitrification allows to fixate the sample at near-native, hydrated conditions while avoiding the formation of crystalline ice, which would damage the sample CITE. When vitrifying, the thickness of the ice is crucial: a thinner layer will results in fewer inelastically scattered electrons during data collection and better SNR. However, too-thin ice can crack and increase issues with preferential orientation or particle distribution.

\begin{figure}[ht]
    \centering
    \includegraphics[width=.5\textwidth]{example-image.png}
    \caption{TODO: figure with summary of sample prep}
    \label{fig:sample_prep}
\end{figure}

\subsubsection{Data collection}
The vitrified sample is loaded into the electron microscope (under cryogenic conditions and vacuum, in order to maintain the sample vitrified and uncontaminated), and suitable positions on the grid are chosen for imaging, prioritizing for: presence of the target protein or complex, fewer contaminations, and lower ice thickness. In modern workflows and software, this step and the subsequent data collection are increasingly automated, allowing for higher throughput (up to several thousand images per day) and lower human intervention CITE.

During collection, the sample stage and electron beam are moved to each position to collect micrographs. At each position, a short movie is recorded, consisting of a few low-dose, short-exposure frames: this allows to reduce motion during frame collection (resulting in blur and therefore noise); the frames will later be aligned and averaged to produce a single higher-contrast image. Direct detectors are therefore crucial not only for their raw improvement in achievable resolution, but also for their high speed, contributing to the reduction of yet another source of noise.

A key component of data collection in cryo-EM is defocus adjustment. The phase contrast generated by electron diffraction depends on spatial frequency and defocus (see \nameref{image_formation}); to ensure all spatial frequencies are sampled, micrographs are typically collected at a range of defoci.

(TODO: I'd like to add something like "There are many steps and challenges in the setup of a data collection; as this thesis does not focus on the experimental side, this part will be treated only superficially" or similar... not sure if/where)

\subsubsection{Image formation and CTF}\label{image_formation}

Defocus, 

TODO: would be cool to link here somehow to my CTF simulator in napari :)

\begin{outline}
\1 Background on Cryo-EM in structural biology
    \2 Base concepts (do not spend too much time on this)
        \3 \tick sample prep
        \3 \tick exposure damage
        \3 \tick role of detectors and resolution revolution \cite{faruqiCCDDetectorsHighresolution2000}
        \3 \tick data collection
        \3 image formation
        \3 CTF
        \3 defocus (and ranges of defocus) and info spread
        \3 dataset cleaning
        \3 classification
        \3 typical processing steps, theory behind reconstruction and classification (more time here as it's more relevant for cryoet and my work)
        \3 single particle analisys
        \3 model building
    \2 typical use cases (some example of single particle data, maybe using ftsz images) and advantages
        \3 no crystal, simpler sample prep
        \3 heterogeneity to a certain extent
        \3 more native state
        \3 bigger targets? complexes
    \2 limitations (leading up to cryo-et) and comparison with other techniques (xray, nmr)
        \3 sample prep/thickness/denaturation
        \3 context missing
        \3 ability to capture dynamics/heterogeneity (more/less than other techniques)
        \3 Signal/Noise
\1 cryo-et, concepts and advantages
    \2 base concepts and differences from SPA from thechnical standpoint
        \2 cryoem, but different data acquisition and thus data processing
        \3 exposure damage and dose issues
        \3 sample prep (thinnes especially important because SNR)
        \3 tilt-series alignment (fiducials), deformation, other aberrations, how they are tackled
        \3 3D CTF
        \3 missing wedge, issues it introduces
        \3 feature deletion (e.g: fiducials)
        \3 segmentation/morphology
        \3 STA (picking, averaging, classification, ...) for non-unique objects (template matching, geometric and template free, machin learning)
    \2 some references/reviews (cool stuff you can do)
        \3 \cite{turkPromiseChallengesCryoelectron2020,lucicCryoelectronTomographyChallenge2013}
    \2 what it provides compared to SPA
        \3 more native state and context preservation
        \3 single-particle insights
        \3 meso-scale and superstructural information
    \2 limitations (both in terms of technique, and in terms of current state of development)
        \3 (some mentioned before, maybe need to connect these two parts a bit better)
        \3 need for custom workflow cause the goal and problems are usually unique
        \3 issues caused by the missing wedge (alignement, reconstruction, interpretability)
        \3 SNR, denoising
        \3 visualisation/interaction
        \3 object identification (crowding
    \2 related techinques to overcome some limitations
        \3 FIB, what it enables and how it works
        \2 CLEM
\end{outline}


