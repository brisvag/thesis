\chapter{Cryo-EM and cryo-ET}

In this chapter, I describe the role of cryo-EM and cryo-ET in the field of structural biology, explain their fundamentals, and discuss advatages and limitations.

\section{Structural biology and X-ray crystallography}
Structural biologists employ several techinques to understand the structure and function of biomolecular systems, typically proteins, nucleic acids and their ligands. Arguably, the three giants in the field are X-ray crystallography, Nuclear Magnetic Resonance (NMR) and cryo-electron microscopy (cryo-EM), with several other techniques playing complementary, more specialized or niche roles (such as mass spectroscopy, neutron scattering or small-angle scattering). Each technique has pros and cons, making it suited to different samples and applications.

In X-Ray crystallography --- in many ways the predecessor to cryo-EM as all-purpose structural biology technique --- crystals are grown from the sample of interest; the crystals are then illuminated by an X-ray beam, and the diffraction pattern thus created can be detected and used to reconstruct the three-dimensional (3D) structure of the sample. Thanks to the short wavelength of X-rays --- as opposed to visible light --- it is possible to localize the positions of atoms with sub-angstrom precision CITE.
X-ray crystallography has played a major role in the development of structural biology, and is still the primary source of protein structures on the Protein Data Bank~\cite{bermanProteinDataBank2000,bermanAnnouncingWorldwideProtein2003}. However, the need for crystallization poses a few limits. The crystallization procedure is often hard to devise and reproduce, extending the time and resources needed for sample preparation. Moreover, the resulting crystal "forces" the sample into a crystalline lattice, often with biologically irrelevant chemical states, restricting the conformational freedom of the molecule and limiting the biological significance of the obtained structures CITE.

\section{Cryo-electron microscopy}

Cryo-EM improves on these aspects, by using vitrification to capture the sample at a near-native state, and by forgoing crystallization and the analysis of diffraction patterns in favor of free single molecules and the use of the central slice theorem to combine many projections into a single 3D structure.

\subsection{Base concepts}

The typical workflow of cryo-em can be summarized as follows:

\begin{enumerate}[noitemsep]
\item sample preparation
\item data collection
\item data processing
\item model building
\end{enumerate}

\paragraph{1. Sample preparation} This is a paragraph about sample preparation

\begin{outline}
\1 Background on Cryo-EM in structural biology
    \2 Base concepts (do not spend too much time on this)
        \3 sample prep
        \3 data collection, exposure damage, role of detectors and resolution revolution \cite{faruqiCCDDetectorsHighresolution2000}
        \3 image formation, CTF
        \3 typical processing steps, theory behind reconstruction and classification (more time here as it's more relevant for cryoet and my work)
        \3 single particle analisys
        \3 model building
    \2 typical use cases (some example of single particle data, maybe using ftsz images) and advantages
        \3 no crystal, simpler sample prep
        \3 heterogeneity to a certain extent
        \3 more native state
        \3 bigger targets? complexes
    \2 limitations (leading up to cryo-et) and comparison with other techniques (xray, nmr)
        \3 sample prep/thickness/denaturation
        \3 context missing
        \3 ability to capture dynamics/heterogeneity (more/less than other techniques)
        \3 Signal/Noise
\1 cryo-et, concepts and advantages
    \2 base concepts and differences from SPA from thechnical standpoint
        \2 cryoem, but different data acquisition and thus data processing
        \3 exposure damage and dose issues
        \3 sample prep (thinnes especially important because SNR)
        \3 tilt-series alignment (fiducials), deformation, other aberrations, how they are tackled
        \3 3D CTF
        \3 missing wedge, issues it introduces
        \3 feature deletion (e.g: fiducials)
        \3 segmentation/morphology
        \3 STA (picking, averaging, classification, ...) for non-unique objects (template matching, geometric and template free, machin learning)
    \2 some references/reviews (cool stuff you can do)
        \3 \cite{turkPromiseChallengesCryoelectron2020,lucicCryoelectronTomographyChallenge2013}
    \2 what it provides compared to SPA
        \3 more native state and context preservation
        \3 single-particle insights
        \3 meso-scale and superstructural information
    \2 limitations (both in terms of technique, and in terms of current state of development)
        \3 (some mentioned before, maybe need to connect these two parts a bit better)
        \3 need for custom workflow cause the goal and problems are usually unique
        \3 issues caused by the missing wedge (alignement, reconstruction, interpretability)
        \3 SNR, denoising
        \3 visualisation/interaction
        \3 object identification (crowding
    \2 related techinques to overcome some limitations
        \3 FIB, what it enables and how it works
        \2 CLEM
\end{outline}


