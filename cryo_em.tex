\chapter{Cryo-EM and cryo-ET}

\begin{outline}
\1 Background on Cryo-EM in structural biology
    \2 Base concepts and uses
        \3 sample prep, image formation, data collection (do not spend too much time on this), role of detectors and resolution revolution \cite{faruqiCCDDetectorsHighresolution2000}
        \3 typical processing steps, theory behind reconstruction and classification (more time here as it's more relevant for cryoet and my work)
    \2 typical use cases (some example of single particle data, maybe using ftsz images)
    \2 limitations (leading up to cryo-et) and comparison with other techniques (xray, nmr)
        \3 sample prep/thickness/denaturation
        \3 context missing
        \3 ability to capture dynamics (more/less than other techniques)
\1 cryo-et, concepts and advantages
    \2 some references/reviews (cool stuff you can do)
        \3 \cite{turkPromiseChallengesCryoelectron2020,lucicCryoelectronTomographyChallenge2013}
    \2 differences from SPA from thechnical standpoint
        \3 sample prep (thinnes especially important)
        \3 exposure damage and dose issues
        \3 tilt-series alignment, deformation, other aberrations
        \3 FIB?
    \2 what it provides compared to SPA
        \3 more native state and context preservation
        \3 single-particle insights
        \3 meso-scale and superstructural information
    \2 limitations (both in terms of technique, and in terms of current state of development)
        \3 (some mentioned before, maybe need to connect these two parts a bit better)
\end{outline}
