\chapter{Cryo-EM and cryo-ET}

\begin{outline}
\1 Background on Cryo-EM in structural biology
    \2 Base concepts (do not spend too much time on this)
        \3 sample prep
        \3 data collection, exposure damage, role of detectors and resolution revolution \cite{faruqiCCDDetectorsHighresolution2000}
        \3 image formation, CTF
        \3 typical processing steps, theory behind reconstruction and classification (more time here as it's more relevant for cryoet and my work)
        \3 single particle analisys
        \3 model building...
    \2 typical use cases (some example of single particle data, maybe using ftsz images) and advantages
        \3 no crystal, simpler sample prep
        \3 heterogeneity to a certain extent
        \3 more native state
        \3 ...
    \2 limitations (leading up to cryo-et) and comparison with other techniques (xray, nmr)
        \3 sample prep/thickness/denaturation
        \3 context missing
        \3 ability to capture dynamics/heterogeneity (more/less than other techniques)
        \3 Signal/Noise
        \3 somewhere (maybe not here?) there should be a part on algorithmic/computational limitations, which are being or could be tackled in the future
\1 cryo-et, concepts and advantages
    \2 base concepts and differences from SPA from thechnical standpoint
        \3 exposure damage and dose issues
        \3 sample prep (thinnes especially important because SNR)
        \3 tilt-series alignment, deformation, other aberrations
        \3 missing wedge
        \3 FIB?
        \3 STA (picking, averaging, classification, ...)
    \2 some references/reviews (cool stuff you can do)
        \3 \cite{turkPromiseChallengesCryoelectron2020,lucicCryoelectronTomographyChallenge2013}
    \2 what it provides compared to SPA
        \3 more native state and context preservation
        \3 single-particle insights
        \3 meso-scale and superstructural information
    \2 limitations (both in terms of technique, and in terms of current state of development)
        \3 (some mentioned before, maybe need to connect these two parts a bit better)
        \3 need for custom workflow cause the goal and problems are usually unique
        \3 issues caused by the missing wedge (alignement, reconstruction, interpretability)
        \3 SNR, denoising
        \3 visualisation/interaction
\end{outline}
