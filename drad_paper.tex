\section{Introduction}
Cell division is a fundamental biological process that allows a single mother cell to produce two daughter cells.
In bacteria, cell division occurs through binary fission~CITE and involves two major steps: (i) inward growth of a new dividing cell wall, known as septum, down the middle of the cell and (ii) separation of the two daughter cells through the action of cell wall hydrolases.
The new septum can result from one of two mechanisms~CITE.
In most Gram-negative bacteria, including \emph{Escherichia coli}, cell division is achieved by progressive constriction of the bacterial cell wall at mid-cell until fusion of cell envelope structures allows the separation of daughter cells.
In contrast, in Gram-positive bacteria, such as \emph{Staphylococcus aureus} or \emph{Bacillus subtilis}, cell division results from progressive synthesis of a new septal cross-wall at mid-cell that advances centripetally from the outer cell wall like a closing iris until full closure of the septal disk~CITE.
Cryo-electron micrographs of various dividing Gram-positive bacteria have revealed that this septum is actually composed of two adjacent cross-walls separated by a low density region~CITE.
In this mode of division, known as septation, the cell diameter of the mother cell is unaffected.
Some bacteria, such as gonococci, have also been observed to divide using a combination of constriction and septation~CITE.
In the constriction mode, daughter cell separation occurs at the same time as the division process, while in the case of septation, splitting of the daughter cells only occurs once a complete, new septum has been synthesized.
This splitting step can be slow and gradual as in \emph{B.
subtilis}~CITE, or instead very fast through a `popping' mechanism as in \emph{S.
aureus} and actinobacteria~CITE.
