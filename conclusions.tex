\chapter{Discussion and future perspectives}\label{future}

This manuscript does not mark the end of the projects herein; the development of blik --- and the ecosystem it lives in --- is ongoing, and much more can be done to further our understanding of \textit{D. radiodurans} and its key players, such as FtsZ and HU.

\localtableofcontents

\section{blik, napari, and }

The development of blik has brought many opportunities

brought lots of good napari stuff
useful for rapid prototyping


\section{\textit{D. radiodurans}: FtsZ, HU, and more}


\section{cryo-et}

what it's good for, what not




\localtableofcontents

\begin{outline}
\1 usefulness of blik in working with data and developing new tools
    \2 especially good here to mention other projects that benefited from this work (either blik or related tools)
\1 new insights into D.rad
\1 evolution of field during thesis, new software, new people... new initiatives (some words on teamtomo maybe should go here as well!)
\1 scipion (need to finish that...)
\1 MD + cryoet = awesome
\end{outline}
