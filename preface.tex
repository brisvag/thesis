\chapter{Preface}

This thesis is the result of interdisciplinary work at the interface of computer science, image processing, structural biology and microbiology.
The primary driving force of this work is the development of open-source, user-friendly and reusable data analysis and visualisation software for the cryo-electron micriscopy community and the scientific image community at large.
The tools and software developed during my thesis stemmed from the need to overcome the obstacles of working with several projects I undertook or contributed to, mostly surrounding cryo-electron tomography data.

Coming from a master's degree specialized in molecular dynamics, I was often faced with the problem of attempting to simulate biological systems, with only the structural knowledge from experiments done in far-from-native conditions, always doubting the effect of initial model bias on my simulations.
Cryo-ET offered an exciting prospect, promising quasi-native structures, \textit{in situ}, at resolutions slowly approaching those of single particle cryo-EM.

My thesis started in October 2020 under the supervision of Irina Gutsche and Joanna Timmins, with the ambitious project of using cryo-ET to understand at the structural level the dynamic and efficient compacting of chromatin in the bacterium \textit{Deinococcus radiodurans}.
During the first year of my PhD, while I learned to work with tomography data.

\begin{outline}
\1 \tick masters -> gateway into structural bio, visualisation and development
\1 Overview of thesis
    \2 initial work on bacterial cromatin
    \2 napari, move onto development > blik
    \2 Dr. project (connection with initial project, interest in septation)
\1 Importance of understanding 3D structure of molecules
    \2 Cryo-em's contribution to this, resolution revolution
    \2 some review, examples of breakthroughs thanks to structure findings (covid mention might be nice)
\1 Importance of context, mesoscale and superstructures
    \2 examples of SPA that cannot give insights due to lack of mesoscale/context
    \2 Cryo-et and its power for this at both lower and higher res
\1 Importance of visualization in interpreting complex data
    \2 hypothesis generation, testing
    \2 challenges due to complexity/dimensionality of data
    \2 need for real-time interaction
    \2 what I set out to do about it (napari, blik)
\1 D. radiodurans
    \2 why it's interesting
        \3 radiation resistance, dna repair
        \3 different from typical (gram+/-)
        \3 relatively simple systems for stuff we were interested in (HU)
    \2 what we hope to learn (DNA repair, protection via wall, etc)
\1 in short, goals of this thesis
    \2 enable people to do powerful tomo on superstructures
    \2 apply these methods on biological system
    \2 Understand mechanisms of Deinococcus
\1 what each chapter is about (either in brief here at the end, or spread out trhough the previous bullet points with references to each chapter)
\1 nature of thesis (open source dev, collaborative work, less focus on single project/structure). Not sure if/how/where to do this, but it feels important to mention
\1 publications, projects, collaborations resulted from this work
\end{outline}
