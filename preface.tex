\chapter{Preface}

This thesis is the result of interdisciplinary work at the interface of computer science, structural biology and image processing. The main focus is on the development of open-source, user-friendly and reusable software for the Cryo-ET [...]

\begin{outline}
\1 masters -> gateway into structural bio, visualisation and development
\1 Overview of thesis
    \2 initial work on bacterial cromatin
    \2 napari, move onto development > blik
    \2 Dr. project (connection with initial project, interest in septation)
\1 Importance of understanding 3D structure of molecules
    \2 Cryo-em's contribution to this, resolution revolution
    \2 some review, examples of breakthroughs thanks to structure findings (covid mention might be nice)
\1 Importance of context, mesoscale and superstructures
    \2 examples of SPA that cannot give insights due to lack of mesoscale/context
    \2 Cryo-et and its power for this at both lower and higher res
\1 Importance of visualization in interpreting complex data
    \2 hypothesis generation, testing
    \2 challenges due to complexity/dimensionality of data
    \2 need for real-time interaction
    \2 what I set out to do about it (napari, blik)
\1 D. radiodurans
    \2 why it's interesting
        \3 radiation resistance, dna repair
        \3 different from typical (gram+/-)
        \3 relatively simple systems for stuff we were interested in (HU)
    \2 what we hope to learn (DNA repair, protection via wall, etc)
\1 in short, goals of this thesis
    \2 enable people to do powerful tomo on superstructures
    \2 apply these methods on biological system
    \2 Understand mechanisms of Deinococcus
\1 what each chapter is about (either in brief here at the end, or spread out trhough the previous bullet points with references to each chapter)
\1 nature of thesis (open source dev, collaborative work, less focus on single project/structure). Not sure if/how/where to do this, but it feels important to mention
\1 publications, projects, collaborations resulted from this work
\end{outline}
