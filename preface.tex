\chapter{Preface}

This thesis is the result of interdisciplinary work at the interface of computer science, image processing, structural biology and microbiology.
The primary driving force of this work is the development of open-source, user-friendly and reusable data analysis and visualization software for the cryo-electron microscopy community and the scientific image community at large.
The tools and software developed during my thesis stemmed from the need to overcome the varied challenges encountered while working on the several structural biology projects I undertook or contributed to, mostly surrounding cryo-electron tomography data.

Coming from a master's degree specialized in molecular dynamics, I was often faced with the problem of attempting to simulate biological systems, with only the structural knowledge from experiments done in not-so-native conditions, always doubting the effect of initial model bias on my simulations.
Cryo-ET offered an exciting prospect, promising quasi-native structures, \textit{in situ}, at resolutions increasingly approaching those of single particle cryo-EM.

My thesis started in October 2020 in the MICA group at the Institute de Biologie Structurale (IBS) in Grenoble, under the supervision of Irina Gutsche and co-supervision of Joanna Timmins, leader of the GenOM team, with the ambitious project of using cryo-ET to understand the dynamic and efficient compacting of chromatin in the bacterium \textit{Deinococcus radiodurans} at the structural level.

During the first year of my PhD, I confronted the problem of trying to pick and process very small proteins in single particle cryo-EM data with nothing but DNA filaments as a guide, while learning the workflows and intricacies of cryo-EM and cryo-ET data processing.
In the first part of this manuscript, I lay out the theoretical foundations of cryo-electron microscopy and single particle analysis in \fullref{em}, as well as tomography and subtomogram averaging in \fullref{et}, contextualized within typical data collection and processing workflows.

After a year of working on bacterial chromatin, my thesis committee and I concluded that this project required more work and careful design on the sample preparation side in order to be feasible in single particle, let alone in cryo-ET.
The goal of my thesis thus shifted: I started contributing to a few other projects within the group, while focusing more intently on the continued development of various software tools I had been working on.

One of such tools was blik, a cryo-ET-oriented plugin for napari, a rapidly growing scientific visualization tool designed for interactivity, ease of use, and customizability.
Blik was born out of the need to not only visualize cryo-ET data in its native 3D space, but also to be able to create custom tools to annotate, select, or otherwise manipulate the data in a way that acted less like a black-box and more like an exploratory experience.
With my growing contributions to napari, I was soon asked to join the core-developers of the project, whose development and maintenance I continued throughout my PhD, and which is now used by hundreds of researchers from several different fields.
The publication that resulted from this work is presented in \fullref{blik}.

While the original thesis project about bacterial chromatin went to the back burner, together with the GenOM group we continued to study \textit{D. radiodurans}, fascinated by its incredible radiation resistance and its odd cellular structure which sets it apart from most bacteria.
In my second year, I spent three months visiting Linda Sandblad's group at the Centre for Electron Microscopy in Umeå, where we collected the first set of tilt-series of the bacterium from FIB-milled samples, revealing a host of new interesting features that we hadn't seen before.
Among many, two captured our interest and became the main focus of my subsequent data analysis work: the distinctive arch-shaped FtsZ --- a protein involved in the bacterial septation process that we were already working on using single particle analysis --- and the variegated cell walls and septa --- already under investigation by GenOM with fluorescence microscopy and other techniques.
A publication about our work on the cell wall and septation mechanism of \textit{D. radiodurans} is underway, and is attached in its current form in \fullref{drad}.

TODO: ftsz....

Several smaller projects and collaborations materialized during this thesis, giving birth to many scripts and tools that may yet benefit others; some of these are presented in \fullref{annexes}. 

TODO Excuse/opportunity to work on development of stuff

TODO: dunno where this next stuff goes...

\begin{outline}
\1 Importance of understanding 3D structure of molecules
    \2 Cryo-em's contribution to this, resolution revolution
    \2 some review, examples of breakthroughs thanks to structure findings (covid mention might be nice)
\1 Importance of context, mesoscale and superstructures
    \2 examples of SPA that cannot give insights due to lack of mesoscale/context
    \2 Cryo-et and its power for this at both lower and higher res
\1 Importance of visualization in interpreting complex data
    \2 hypothesis generation, testing
    \2 challenges due to complexity/dimensionality of data
    \2 need for real-time interaction
    \2 what I set out to do about it (napari, blik)
\1 D. radiodurans
    \2 why it's interesting
        \3 radiation resistance, dna repair
        \3 different from typical (gram+/-)
        \3 relatively simple systems for stuff we were interested in (HU)
    \2 what we hope to learn (DNA repair, protection via wall, etc)
\1 in short, goals of this thesis
    \2 enable people to do powerful tomo on superstructures
    \2 apply these methods on biological system
    \2 Understand mechanisms of Deinococcus
\1 what each chapter is about (either in brief here at the end, or spread out trhough the previous bullet points with references to each chapter)
\1 nature of thesis (open source dev, collaborative work, less focus on single project/structure). Not sure if/how/where to do this, but it feels important to mention
\1 publications, projects, collaborations resulted from this work
\end{outline}
