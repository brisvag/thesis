\chapter{Preface}

This thesis is the result of interdisciplinary work at the interface of computer science, image processing, structural biology and microbiology.
The primary driving force of this work has been the development of open-source, user-friendly and reusable data analysis and visualization software for the cryo-electron microscopy community and the scientific image community at large.
The tools and software developed during my thesis stemmed from the need to overcome the various challenges encountered while working on the several structural biology projects I undertook or contributed to, mostly involving cryo-electron tomography data.

Coming from a master's degree specialized in molecular dynamics, I was often faced with the problem of attempting to simulate biological systems, with only the structural knowledge from experiments done in not-so-native conditions, always doubting the effect of initial model bias on my simulations.
Cryo-ET offered an exciting prospect, promising quasi-native structures, \textit{in situ}, at resolutions increasingly approaching those of single particle cryo-EM.

My thesis started in October 2020 in the MICA group at the Institute de Biologie Structurale (IBS) in Grenoble, under the supervision of Irina Gutsche and co-supervision of Joanna Timmins, leader of the GenOM team, with the ambitious project of developing cryo-ET visualization and analysis tools with the goal of understanding the 3D structural organization of the highly compact and yet dynamic chromatin of the radiation resistant bacterium \textit{Deinococcus radiodurans}.

During the first year of my PhD, I realised that the chromatin model we planned to investigate was actually two-dimensional, due to its strong absorbtion to the air-water interface.
Thus, I ended up using single particle cryo-EM instead of cryo-ET, whereby I confronted the problem of trying to pick and process very small proteins (the nucleoid associated protein HU) in single particle cryo-EM data collected by my supervisors, with nothing but DNA filaments as a guide, while learning the workflows and intricacies of cryo-EM and cryo-ET data processing.
In the first part of this manuscript, I lay out the theoretical foundations of cryo-electron microscopy and single particle analysis (\fullref{em}), as well as tomography and subtomogram averaging (\fullref{et}), contextualized within typical data collection and processing workflows.

After a year of working on this 2D model of \textit{Deinococcus radiodurans} chromatin, my thesis committee and I concluded that this project required more work and careful redesign of the sample preparation strategy in order to be feasible in single particle, let alone in cryo-ET.
The goal of my thesis thus shifted to new biological targets in \textit{D. radiodurans}, while continuing my work on the development of software tools for cryo-ET.

One of such tools was blik, a cryo-ET-oriented plugin for napari, a rapidly growing scientific visualization tool designed for interactivity, ease of use, and customizability.
We started working on blik --- together with Alister Burt, another PhD student in MICA in his last year --- out of the need to not only visualize cryo-ET data in its native 3D space, but also to be able to create custom tools to annotate, select, or otherwise manipulate the data in a way that acted less like a black-box and more like an exploratory experience.
In particular, blik is aimed at the study of superstructural complexes --- such as membrane lattices and filaments --- whose investigation is often the reason why researchers turn to cryo-ET.
With my growing contributions to napari, I was soon asked to join the core-developers of the project, whose development and maintenance I continued throughout my PhD, and which is now used by hundreds of researchers from several different fields.
The work on blik resulted in a publication, which is included and contextualized within this thesis (\fullref{blik}).

While the original thesis project about bacterial chromatin went to the back burner, together with the GenOM group we continued to study \textit{D. radiodurans}, fascinated by its incredible radiation resistance and its odd cellular structure which sets it apart from most bacteria.
In my second year, I spent three months with my supervisor visiting Linda Sandblad's group at the Centre for Electron Microscopy in Umeå, where Irina collected the first set of tomograms of the bacterium from cryo-FIB-milled samples, revealing a host of new interesting features that we hadn't seen before.
Among many, two captured our interest and became the main focus of my subsequent data analysis work: the distinctive arch-shaped FtsZ --- a protein involved in bacterial septation that we were already working on using single particle analysis --- and the multi-layered cell walls and septa --- already under investigation by the GenOM team with fluorescence microscopy and other techniques.
A publication about our work on the cell wall and septation mechanism of \textit{D. radiodurans} is underway, and is attached to this manuscript in its current form (\fullref{drad}).

Throughout the years of my PhD, I worked on several other projects worthy of inclusion in this manuscript.
This includes my original PhD project on HU and bacterial chromatin, the ongoing work on \textit{D. radiodurans} FtsZ combining the use of single particle cryo-EM on purified samples and of subtomogram averaging on the aforementioned tomograms, and a few computational tools I developed along the way (\fullref{other_projects}).

In the final chapters of this thesis, I discuss the state of the cryo-ET software ecosystem in academia (\fullref{software}), and the future of the projects in this thesis and of cryo-ET as a whole (\fullref{future}).
