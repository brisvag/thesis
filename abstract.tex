\section*{Abstract}
\addcontentsline{toc}{chapter}{Abstract}  % this combined with section*s makes sure tocs are good

\begin{outline}
\1 Cryo-et is powerful structural bio technique: enables in situ, context-aware, mesoscale insights
\1 Software is not yet mature and fragmented, often limits researchers
\1 visualisation, interactivity, hackability are iportant
\1 developed blik and tools as part of this thesis to address the above problems
    \2 probably briefly mention napari/hackability/modularity...
\1 used them on drad to understand septation mechanism and ftsz
    \2 wall profiles and changes over cell cycle
    \2 mesosomes (probably not worth mentioning here)
    \2 ftsz (and brief mention of SPA as well?)
\1 mention here papers included in the thesis and what they are about? Or is this just Preface material?
\end{outline}


Since its inception, cryo-electron microscopy (cryo-EM) has shown its potential for high resolution structural biology

played a key role in the field of structural biology.

% Random chatgpt draft
%
% Cryo-electron microscopy (cryo-EM) and cryo-electron tomography (cryo-ET) are transformative technologies in structural biology, enabling high-resolution visualization of biological macromolecules and cellular structures. This thesis explores the development and application of novel interactive visualization software, Blik, designed to enhance the analysis and annotation of cryo-ET data. Additionally, it investigates the septation mechanism in the extremophilic bacterium Deinococcus radiodurans using both cryo-EM and cryo-ET.
%
% The first section provides an extensive review of cryo-EM and cryo-ET methodologies, highlighting recent advancements and ongoing challenges. The need for advanced visualization tools is addressed through the development of Blik, an innovative software solution that facilitates interactive data analysis, annotation, and particle picking in cryo-ET datasets. This tool is critically evaluated against existing software, demonstrating superior performance in usability and accuracy through user feedback and case studies.
%
% The biological focus of this thesis is the study of D. radiodurans, known for its extraordinary resistance to DNA-damaging conditions. Using cryo-EM and cryo-ET, this research elucidates the cellular structures and mechanisms underpinning its robust septation process. Detailed methodologies for sample preparation, imaging, and data processing are described, ensuring reproducibility and transparency in the experimental approach.
%
% Two key papers are incorporated into the thesis: the first detailing the development and capabilities of Blik, and the second presenting new insights into the septation mechanism of D. radiodurans. These papers are contextualized within the broader scope of the thesis, emphasizing their contributions to the fields of structural biology and microbial physiology.
%
% The discussion synthesizes findings from both the technological and biological investigations, proposing future research directions and potential advancements in cryo-EM and cryo-ET techniques. Limitations encountered during the research are critically examined, with suggestions for overcoming these challenges in subsequent studies.
%
% In conclusion, this thesis presents significant contributions to both the development of interactive visualization tools and the understanding of bacterial septation mechanisms. Blik stands as a valuable resource for researchers in cryo-ET, and the insights into D. radiodurans offer promising avenues for further microbiological research. This work underscores the interplay between technological innovation and biological discovery, fostering advancements in the visualization and interpretation of complex biological systems.
%

\paragraph{Keywords}
Cryo-electron tomography, ...

\newpage
\section*{Résumé}

Le même mais en français.

\paragraph{Mots clés}
Tomographie cryoélectronique, ...
