\section*{Abstract}
\addcontentsline{toc}{chapter}{Abstract}  % this combined with section*s makes sure tocs are good

Cryo-electron tomography (cryo-ET) is an imaging technique that allows to reconstruct the three-dimensional volume of whole biological samples --- such as cells and tissues --- \textit{in situ}, in near-native conditions and within their biological context.
With the development of direct electron detectors and increased computing power, in recent years cryo-ET has rapidly become a powerful and ubiquitous tool in the arsenals of structural biologists, sometimes capable of reaching sub-nanometer resolution.
Due to the yet immature software ecosystem and the complexity of the workflows, working with cryo-ET data is still error-prone and requires constant human supervision, especially when it comes to subtomogram averaging pipelines attempting to reach high resolution reconstructions.
This thesis presents the development of blik, a tool aimed at simplifying working with cryo-ET data by providing interactive visualisation, analysis and annotation that are both powerful and customizable, with a focus on the analysis of superstructural systems such as filaments and membranes.
Using blik and other tools, we investigate the cell wall composition and septation mechanism in the radiation resistant bacterium \textit{Deinococcus radiodurans}, and study the structure and role of FtsZ, a tubulin homologue which plays a key role in cell division.

\subsection*{Keywords}
Cryo-electron tomography, superstructure, blik, \textit{D. radiodurans}, FtsZ

\newpage

\begin{otherlanguage}{french}

\section*{Résumé}
La tomographie cryo-électronique (cryo-ET) est une technique d'imagerie qui permet de reconstruire le volume tridimensionnel d'échantillons biologiques entiers --- tels que les cellules et les tissus --- \textit{in situ}, dans des conditions proches de l'état natif et dans leur contexte biologique.
Avec le développement de détecteurs directs d'électrons et l'augmentation de la puissance de calcul des ordinateurs, la cryo-ET est rapidement devenue, ces dernières années, un outil puissant et omniprésent dans l'arsenal des biologistes structuraux, parfois capable d'atteindre une résolution sous-nanomètrique.
En raison de l'écosystème logiciel encore immature et de la complexité des flux de travail, travailler avec des données cryo-ET est encore sujet à des erreurs et nécessite une supervision humaine constante, en particulier lorsqu'il s'agit de pipelines de moyennation de sous-tomogrammes qui visent à atteindre des reconstructions à haute résolution.
Cette thèse présente le développement de blik, un outil conçu pour simplifier le travail sur les données cryo-ET en fournissant une visualisation interactive, une analyse et une annotation qui sont à la fois puissantes et personnalisables, en se concentrant sur l'analyse d'assemblages supramoléculaires tels que les filaments et les membranes.
En utilisant blik et d'autres outils, nous examinons la composition de la paroi cellulaire et le mécanisme de septation chez \textit{Deinococcus radiodurans}, une bactérie résistante particulièrement aux radiations , et nous étudions la structure et le rôle de FtsZ, un homologue de la tubuline qui joue un rôle clé dans la division cellulaire.

\subsection*{Mots clés}
Tomographie cryo-électronique, superstructure, blik, \textit{D. radiodurans}, FtsZ
    
\end{otherlanguage}
