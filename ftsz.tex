\chapter{FtsZ: function and structure}\label{ftsz}

As mentioned in \fullref{drad_ftsz}, FtsZ is an extremely well conserved prokaryotic tubulin homologue, known to form ring-like structures (the Z-ring) at the septation site in most bacteria.
It polymerizes in a GTP-dependent fashion to form filaments and bundles, anchoring to the membrane via its partner FtsA, and interacting with several other partners in the cytokinesis and peptidoglycan (PG) synthesis machinery.

Based on crystal structures, filaments have been shown to form through head-to-tail stacking of monomers CITE.
However, there is no structural information regarding the flexible C-terminal region, which is responsible for regulating its activity and interaction with its cellular partners, including FtsA.
It's also unclear how multiple filaments may assemble to form bundles at the structural level and even less so in vivo.
Thus --- while its key role in cell division is undisputed --- the exact mechanism and function of FtsZ are still unclear.
There is no consensus model, as the current evidence is inconclusive, sometimes presenting significant variations between species --- likely ascribable to different shapes or membrane compositions between bacteria~\cite{barrowsFtsZDynamicsBacterial2021,mcquillenInsightsStructureFunction2020}.

\begin{figure}[ht]
    \centering
    \includegraphics[width=.5\textwidth]{example-image.png}
    \caption{TODO: ftsz ring and cystal structure?}
    \label{fig:ftsz_ring}
\end{figure}

\localtableofcontents

\section{State of the art}

Some studies have shown that FtsZ presents mechanical functions that may drive the septation process.
\textit{T. maritima} FtsZ+FtsA expressed in liposomes was shown to form coil-like structures, which can constrict the membrane via a "filament sliding" mechanism, creating a partial septum~\cite{szwedziakArchitectureRingFormed2014}.
In \textit{C. crescentus}, FtsZ may form short, loosely bundled filaments, which may drive constriction via an "iterative pinching" mechanism where repeated rounds of phosphorylation lightly bend the filament, thus slowly pinching the membrane~\cite{liStructureFtsZFilaments2007}.

Other publications investigated the recruitment and signalling role of FtsZ for downstream machinery such as PG synthesis.
FtsZ was shown to undergo plus-end polymerization and minus-end depolymerization, in a GTP-regulated process typical of cytoskeletal proteins called treadmilling~\cite{looseBacterialCellDivision2014}.
In \textit{B. subtilis}, this treadmilling is required to drive the movement along the septum of the PG synthesis centers~\cite{bisson-filhoTreadmillingFtsZFilaments2017}.
A "diffusion-and-capture" model was proposed, where the FtsZ ring performs a recruitment role by engaging in weak transient interactions with downstream machinery for PG synthesis~\cite{baranovaDiffusionCapturePermits2020}.
However, in \textit{S. aureus}, cytokinesis may actually occur in two separate steps: a slow, FtsZ-driven, treadmilling-dependent step which causes initial invagination, followed by a faster step where PG synthesis is the driving force for septation~\cite{monteiroPeptidoglycanSynthesisDrives2018}.

While FtsZ ring formation and PG synthesis are clearly linked, their precise interaction may differ between bacteria.
In \textit{E. coli}, GTP regulates FtsZ treadmilling, which in turn controls the movement of the synthesis machinery~\cite{yangGTPaseActivityCoupled2017}.
However, it does not appear to affect the rate of PG synthesis, as opposed to what happens in \textit{B. subtilis}, which suggest that the presence or absence of an outer membrane may change PG synthesis machinery regulation~\cite{yangGTPaseActivityCoupled2017}.
Indeed, in \textit{E. coli}, some additional proteins may help with a timely invagination of the outer membrane, although they are not needed for septation to occur~\cite{gerdingTransenvelopeTolPal2007}.
This is intriguing for \textit{D. radiodurans} which, despite staining gram-positive, presents an outer membrane.

The disordered C-terminal domain of FtsZ was found the be necessary both for the anchorage to the membrane via FtsA, and to regulate oligomerization as well as bundling with neighboring filaments~\cite{barrowsFtsZDynamicsBacterial2021}.

Collectively, the literature hints that FtsZ treadmilling is likely not the only factor that controls the dynamics of FtsZ and the Z-ring, and that variations across species may be explained by different divergent mechanisms, or some underlying behavior not yet discovered~\cite{barrowsFtsZDynamicsBacterial2021}.

TODO: question, in ~\cite{mcquillenInsightsStructureFunction2020}, it says that the Z-ring is a torus 80-100nm wide, 13-16nm below the membrane. Does this make sense with our data? --- the arc seems to be about 50 nm wide in good cases, but the distance from the membrane makes sense. Also, not sure if "torus" makes sense, because it's flat? Maybe mention after

TODO: question: what about the open conformation thing?

\section{Structural information}

There are a number of structural studies on FtsZ and its partners, using a variety of techniques and model organisms.

In its monomeric form, FtsZ has been reconstructed in X-ray crystallography and cryo-em...

A high resolution (\sim\qty{3}{\angstrom}) structure of the \textit{Klebsiella pneumoniae} FtsZ filament obtained with SPA was deposited by ~\citet{fujitaStructuresFtsZSingle2023}.
Unfortunately, their sample suffered from a strong preferential orientation, which resulted in a significant resolution anisotropicity, hindering the ability to build a model.

TODO...

\subsection{SPA}

Strong pref. orientation, map doesn't look better/worse than ~\citet{fujitaStructuresFtsZSingle2023}.
Issue with seeding angles... angular

Problem of preferential orientation... solution I came up with. No work. stuff.

work so far, classes

detergent + other tricks = now better angular distribution

width

\subsection{STA}

(TODO: already in paper, no need to talk about this probably. other papers used tomography to look into D.rad and also saw what they called ftsz and ftsa ~\cite{sextonSuperresolutionConfocalCryoCLEM2022})

also pref orientation

mention distance between filaments in bundles in SPA vs what we see in Tomos: compatible?

\subsection{Discussion}

Ice thickness should be negligible if we do proper CTF and motion correction (iteratively) ~\cite{aiyerOvercomingResolutionAttenuation2024}.

\begin{outline}
\1 Background on FtsZ
    \2 various structures and how the align with models
\1 structural work
    \2 our work on SPA and filaments
        \3 technical difficulties: where to expand? (pref. orientation, filament picking, angles, classification, 3D rec...)
    \2 how this fits in with the rest
    \2 tomography + fluo from paper (depending on how much we already talk about this there)

\1 future perspective (probably go in discussion)
    \2 new data acquisitions
    \2 solving pref orientation
    \2 ideas for picking
\end{outline}
