% different stuff if printer-friendly or not
\RequirePackage{xifthen} \newboolean{printer-friendly} \setboolean{printer-friendly}{false}

% TODO: move to memoir (requires lots of fixes)
\ifthenelse{\boolean{printer-friendly}}{
    \documentclass[12pt,a4paper,twoside]{memoir} % extbook to allow smaller font
}{
    \documentclass[12pt,a4paper,oneside]{memoir}
}

% general must-have or good-to-have stuff
\usepackage[main=english,french]{babel}
\usepackage{microtype} % fixes some over/under-full box warnings
\usepackage[T1]{fontenc} % better behaviour for fonts (copy-pasting, etc)
\usepackage{XCharter} % font typeface (compared to charter has support for unicode input/output)
\usepackage[svgnames]{xcolor} % colors by name and modifiers to alpha and such
\OnehalfSpacing

% bibliography
\usepackage[super,sort&compress,comma]{natbib}
\bibliographystyle{unsrtnat}
\def\bibfont{\footnotesize}

% hyperrefs and friends
\usepackage[pagebackref,breaklinks,colorlinks,linktoc=all,allcolors=FireBrick]{hyperref}
\urlstyle{same} % don't use monospace font for urls
\renewcommand*\backref[1]{\ifx#1\relax \else (pg. #1) \fi} % backref to citation location formatted as: (pg. N)
\usepackage{hypernat} % fixes backref of ranges. must be after natbib and hyperref
% when in printer-friendly convert href (which normally hides url with text) to a footnote so you can see it
\ifthenelse{\boolean{printer-friendly}}{
    \let\oldhref\href \renewcommand\href[2]{\oldhref{#1}{#2}\footnote{\url{#1}}}
}{}
% hyperlink DOIs (only non-print or it will become a useless footnote)
\ifthenelse{\boolean{printer-friendly}}{}{
    \newcommand*{\doi}[1]{\href{https://doi.org/\detokenize{#1}}{DOI:~\detokenize{#1}}}
}
% better autoref names for chapters and sections
\addto\extrasenglish{
    \def\partautorefname{Part}
    \def\chapterautorefname{Chapter}
    \def\sectionautorefname{Section}
    \def\subsectionautorefname{\sectionautorefname}
    \def\subsubsectionautorefname{\sectionautorefname}
}
\setcounter{secnumdepth}{3} % number down to subsubsection
% "fullref" (especially for printed cause link can't be clicked and it's easier to find with section numbers)
\newcommand*\fullref[1]{\hyperref[{#1}]{\bfseries\autoref*{#1}: \nameref*{#1}}}
% "figref" same as autoref but bold and has optional argument for extra details
% example: \figref[A][SI ]{fig1label} => "SI Figure 1A", all a link
\usepackage{xparse} % for commands with multiple opt arguments
\NewDocumentCommand{\figref}{ O{} O{} m }{\hyperref[{#3}]{\bfseries#2\autoref*{#3}#1}}

% useful miscellaneous
\usepackage{pdfpages} % include pre-rendered pdfs
\usepackage{tocbibind} % add lof and lot to toc
\usepackage{etoc} % local TOCs per-chapter
\renewcommand*{\cftdotsep}{1} \setpnumwidth{3em} \setrmarg{4em} % fit 3-digit numbers in TOCs
\usepackage{bookmark} % allow to end parts/sections (\startatroot}
\usepackage{booktabs} % improved tables
\usepackage{tabularx} % improved tables
\usepackage[section]{placeins} % float barriers with \FloatBarrier
\usepackage{enumitem} % remove item separation with [noitemsep]
\usepackage[perpage, symbol*]{footmisc} % footnotes counters use symbols and reset each page
\usepackage{amsmath,amssymb} % for better math symbols
\usepackage{epigraph} \setlength{\epigraphwidth}{.7\textwidth} \renewcommand\textflush{flushright} % cool quotes
\usepackage[shortcuts]{extdash} % use \-/ for hyphen that allows breaks in words

% figures and captions
\usepackage{graphicx} \graphicspath{{img/}}
\usepackage[format=plain,labelfont={sc,bf},labelsep=endash,textfont={color=black!80},font=footnotesize]{caption}
\usepackage{subcaption} % subfigures
% smart caption that uses TOC title to add in bold at beginning, if given
\newcommand\titledcaption[2][]{
    \ifthenelse{\isempty{#1}}
        {\caption{#2}}
        {\caption[#1]{\textbf{#1.} #2}}
}
% capitalize subfigure lettering
\renewcommand{\thesubfigure}{\Alph{subfigure}}
% force floats to the top when on float-only pages (https://tex.stackexchange.com/a/28565)
\makeatletter
\setlength{\@fptop}{0pt}
\setlength{\@fpbot}{0pt plus 1fil}
\makeatother

% code boxes and frames
\usepackage{minted} % syntax highlighting
\usepackage[nobreak]{mdframed} % framing minted boxes
\surroundwithmdframed[backgroundcolor=black!10]{minted}
% \setminted[python]{linenos} % enable line numbers on all python boxes

% other (order matters...)
\usepackage[autostyle,english=american]{csquotes} \MakeOuterQuote{"} % normal-er quotes (must be after minted)

% situational but useful or cool
% \usepackage{xspace} % smart spaces after commands % \newcommand*{\something}{\textit{SomEthing}\xspace}
% \usepackage{titling} % multiple/different title pages
% \usepackage[blocks]{authblk} \renewcommand\Affilfont{\scriptsize} % multiple authors with affiliations (must import after titling if it's there)
% \usepackage{tcolorbox} % colored titled box (e.g: "NOTE: ...")

% temporary packages
% \usepackage{outlines} % easy nested lists
% \usepackage{pifont} \newcommand{\tick}{\ding{51}} % icons (e.g checkmark)

% tikz stuff
\usepackage{tikz} \usetikzlibrary{positioning,shapes.geometric,arrows,fit}
\tikzstyle{node} = [
    rectangle,
    rounded corners,
    minimum width=15ex,
    minimum height=6ex,
    text centered,
    draw=black,
    fill=brown!20,
    scale=.7,
]
\tikzstyle{opt} = [node, ellipse, dashed]
\tikzstyle{arrow} = [thick,->,>=stealth]
