\chapter{napari: powerful interactive visualisation}

This thesis was heavily shaped by my extended collaboration with the napari project~\cite{thenaparicommunityNapariMultidimensionalImage2024} and its core developers and community~\cite{thenaparicommunityCommunityNapari2024}, whose values and goals for the development of scientific software strongly align with mine.

Our committed effort to bridge many different imaging field to share knowledge and resources, 

\href{https://napari.org/}{napari.org}
This is a test for some url: \url{https://napari.org/}.

\begin{outline}
\1 value proposition
    \2 python, scientific ecosystem (ease of use, integration, customisation)
    \2 numpy, pandas, widespread adoption. Ease of coding and performance with little knowledge
    \2 interactive GUI (human-in-the-loop)
    \2 domain-agnostic: sharing ideas and code between imaging fields (fluo, cryo, astronomy, computer vision, ai...)
\1 development
    \2 community
        \3 recognize specific contributions from others that enabled me
    \2 my contributions as part of the thesis (didn't go in depth for blik paper)
        \3 architecture
        \3 slicing
        \3 vispy rendering
        \3 point/volume interactivity
        \3 plugins
\end{outline}
