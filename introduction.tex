\chapter{Introduction}\label{introduction}

Keep introduction brief (couple of pages). I prefer the separation in chapters rather than in sections inside a big Introduction chapter.

\begin{outline}
\1 Importance of understanding 3D structure of molecules
    \2 Cryo-em's contribution to this, resolution revolution
    \2 some review, examples of breakthroughs thanks to structure findings (covid mention might be nice)
\1 Importance of context, mesoscale and superstructures
    \2 examples of SPA that cannot give insights due to lack of mesoscale/context
    \2 Cryo-et and its power for this at both lower and higher res
\1 Importance of visualization in interpreting complex data
    \2 hypothesis generation, testing
    \2 challenges due to complexity/dimensionality of data
    \2 need for real-time interaction
    \2 what I set out to do about it (napari, blik)
\1 D. radiodurans
    \2 why it's interesting
        \3 radiation resistance, dna repair
        \3 different from typical (gram+/-)
        \3 relatively simple systems for stuff we were interested in (HU)
    \2 what we hope to learn (DNA repair, protection via wall, etc)
\1 in short, goals of this thesis
    \2 enable people to do powerful tomo on superstructures
    \2 apply these methods on biological system
    \2 Understand mechanisms of Deinococcus
\1 what each chapter is about (either in brief here at the end, or spread out trhough the previous bullet points with references to each chapter)
\end{outline}
